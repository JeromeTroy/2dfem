\documentclass{article}

% macros file
%% packages

% bibliography
\usepackage{cite}

% fonts
\usepackage[T1]{fontenc}
\usepackage[utf8]{inputenc}

% ams stuff
\usepackage{amsthm, amsmath, amssymb, amsfonts}

% bold math
\usepackage{bbold, bm, bbm}

% various letters for enumerate fields
\usepackage{enumitem}

% if statments
\usepackage{xifthen}

% graphics
\usepackage{graphicx, float}

% making figures in tex
\usepackage{tikz, pgfplots}
    \pgfplotsset{compat=1.9}   

%% math commands

% sign of value
\newcommand{\sgn}[1]{\mathrm{sgn}\left( #1 \right)}

% function support
\newcommand{\supp}[1]{\mathrm{supp}\left( #1 \right)}

% function restriction
\newcommand{\restrict}[2]{\left. #1 \right|_{ #2 } }

% vectors and vector spaces
\renewcommand{\vec}[1]{\bm{#1}}
\newcommand{\norm}[1]{\left|\left| #1 \right|\right|}
\newcommand{\inner}[1]{\left\langle #1 \right\rangle}

% special functions
\renewcommand{\exp}[1]{\text{exp}\left(#1\right)}
\renewcommand{\log}[1]{
		\mathrm{log}\left( #1 \right)
}

\renewcommand{\sin}[2][1]{
    \mathrm{sin}^{
        \ifthenelse{\equal{#1}{1}}{}{\,#1}
    }
    \left( #2 \right)
}
\renewcommand{\cos}[2][1]{
    \mathrm{cos}^{
        \ifthenelse{\equal{#1}{1}}{}{\,#1}
    }
    \left( #2 \right)
}
\renewcommand{\cot}[1]{\mathrm{cot}\left( #1 \right)}
\renewcommand{\cosh}[2][1]{
    \mathrm{cosh}^{
        \ifthenelse{\equal{#1}{1}}{}{\,#1}
    }
    \left( #2 \right)
}
\renewcommand{\sinh}[2][1]{
    \mathrm{sinh}^{
        \ifthenelse{\equal{#1}{1}}{}{\,#1}
    }
    \left( #2 \right)
}
\renewcommand{\tanh}[1]{
		\mathrm{tanh}\left( #1 \right)
}
\renewcommand{\tan}[1]{
		\mathrm{tan}\left( #1 \right)
}
\newcommand{\sech}[2][1]{
		\mathrm{sech}^{
				\ifthenelse{\equal{#1}{1}}{}{\,#1}
		}
		\left( #2 \right)
}
\newcommand{\asin}[1]{
		\mathrm{arcsin}\left( #1 \right)
}
\newcommand{\acos}[1]{
		\mathrm{arccos}\left( #1 \right)
}
\newcommand{\atan}[1]{
		\mathrm{arctan}\left( #1 \right)
}





% sets
\newcommand{\R}{\mathbb{R}}
\newcommand{\C}{\mathbb{C}}
\newcommand{\N}{\mathbb{N}}
\newcommand{\Z}{\mathbb{Z}}

% derivatives
\usepackage{xifthen}

\renewcommand{\vec}[1]{\mathbf{#1}}

\newcommand{\der}[3][1]{
	\frac{{\text{d}^{
		\ifthenelse{\equal{#1}{1}}{}{\,#1}
	}
	{#2} }}
	{\text{d} {#3}^{
		\ifthenelse{\equal{#1}{1}}{}{#1}
	}
	}
}

\newcommand{\pder}[3][1]{
	\frac{\partial^{
		\ifthenelse{\equal{#1}{1}}{}{\,#1}
	}
	{#2}}
	{\partial {#3}^{
		\ifthenelse{\equal{#1}{1}}{}{#1}
	}
	}
}

% finite element stuff
\newcommand{\seminorm}[1]{\left\lfloor #1 \right\rfloor}


% functional stuff
\newcommand{\embeds}{\hookrightarrow}

% probability
\newcommand{\E}[2][1]{
		\mathbb E_{
				\ifthenelse{\equal{#1}{1}}{}{#1}
		} \left[ #2 \right]
}
\newcommand{\var}[1]{
		\mathrm{Var}\left( #1 \right)
}
\newcommand{\relent}[2]{
		D_\mathrm{KL} \left(
		#1 \mid \mid #2 \right)
}


% page setup
\everymath{\displaystyle}
\usepackage[margin=1in]{geometry}

\textheight 9in
\oddsidemargin 0pt
\evensidemargin \oddsidemargin
\marginparwidth 0.5in
\textwidth 6.5in 

\setlength{\parindent}{0pt}

\newcommand{\femtitle}[1]{
		\textbf{ University of Delaware \\
		Department of Mathematical Sciences \\
		Math 838 : Finite Element and Boundary Element Methods \\
		Jerome Troy \\
		\vskip 0.25in
		Homework #1}
}



\title{Basic Finite Elements}
\author{Jerome Troy}
\date{March 23, 2021}

\begin{document}
  
\maketitle

\section{The Problem}

Let $\Omega \subset \R^2$ be a compact polygonal domain.  
Let $\Gamma_N, \Gamma_D \subset \partial \Omega$ with
\[
  \Gamma_N \cap \Gamma_D = \emptyset, \quad 
  \Gamma_N \cup \Gamma_D = \partial \Omega
.\] 
Let $c \in \R$ be fixed and 
$f : \Omega \to \R$ be a continuous function.
The problem in question is
\begin{gather}
  -\nabla^2 u + c u = f \quad x \in \Omega \\
  \restrict{u}{\Gamma_D} = u_D \quad x \in \Gamma_D \\
  \restrict{\pder{u}{n}}{\Gamma_N} = g \quad x \in \Gamma_N
.\end{gather}
The goal is to solve this using the finite element method.


\section{Triangular Mesh}

To do this, we first partition $\Omega$ into a triangular mesh.  
Let $\mathcal T = \{K_\ell\}_{\ell=0}^{N-1}$ where
\[
  K_\ell^\circ \cap K_j^\circ = \emptyset, \quad 
  \bigcup_{\ell=0}^{N-1} K_\ell = \Omega
.\] 
Each $K_\ell$ is a triangle.  

As a reference let 
$\hat K$ be the triangle with vertices
\[
		\hat{\vec z_0} = \vec 0, \quad 
		\hat{\vec z_1} = (1, 0)^T, \quad 
		\hat{\vec z_2} = (0, 1)^T
.\] 
Then for each $\ell$ we define an affine transformation 
of the form
\[
		\vec F_\ell : \hat K \to K_\ell, \quad 
		\hat{\vec x} \to B_\ell \hat{\vec x} + \vec z^{(\ell)}_0
.\] 
Where
\[
  B_\ell = 
  \begin{bmatrix}
		  x^{(\ell)}_1 - x^{(\ell)}_0 & x^{(\ell)}_2 - x^{(\ell)}_0 \\
		  y^{(\ell)}_1 - y^{(\ell)}_0 & y^{(\ell)}_2 - y^{(\ell)}_0
  \end{bmatrix}
.\] 
In this way, $K_\ell$ has vertices 
$\vec z^{(\ell)}_{0, 1, 2}$ and the affine transformation has
\[
		\vec F_\ell(\hat{\vec z_j}) = \vec z^{(\ell)}_j
.\] 
Thus for any triangle $K_\ell$ in the triangulation, we can
always map back to the original reference triangle $\hat K$.

\subsection{Continuous Piecewise Linear Functions}

A finite element is of the form $(K, \mathcal P, \mathcal N)$.
For this program, we will use $K = K_\ell$, then
$\mathcal P$ will be the set of all linear polynomials defined on $K$,
and $\mathcal N$ will be the degrees of freedom defined by
\[
		\mathcal N = \{N_0, N_1, N_2\}, \quad 
		N_i(f) = f(\vec z_i)
.\] 
Using affine equivalence, these can be mapped back to the reference element:
$(\hat K, \hat{\mathcal P}, \hat{\mathcal N})$.  
Then $\hat{\mathcal N}$ induces a dual basis for $\hat{\mathcal P}$ of 
the form
\begin{gather}
		\hat{\mathcal B} = \{\hat \varphi_0, \hat \varphi_1, \hat \varphi_2\}
		\\
		\hat \varphi_0(x, y) = 1 - x - y \\
		\hat \varphi_1 = x \\
		\hat \varphi_2 = y
\end{gather}
Using the affine transformation, we can map this basis on the 
reference triangle to that of any triangle using
\[
		\varphi^{(\ell)}_i(\vec x) = 
		\hat \varphi_i(\vec F^{-1}_\ell(\vec x))
.\] 

These basis functions can then be assembled together to build global 
basis functions.  The global basis functions will be 
$\{\phi_\ell\}_{\ell=0}^{N-1}$.  Each of which are defined on 
all of $\Omega$.  These are designed so that 
\[
		\phi_\ell(\vec z_j) = \delta_{\ell j}
.\] 
Point evaluation of these functions is a bit complex, but proceeds as 
follows.  
Given $\vec x \in \Omega$, let $0 \leq \ell \leq N - 1$ so that 
$\vec x \in K_\ell$.  Then if 
$K_\ell \subset \supp \phi_i$ we have an index $k_{\ell i}$ which
indicates for triangle $K_\ell$ that the node $\vec z_i$ occupies
vertex $\vec z_{k_{\ell i}}^{(\ell)}$.
Point evaluation is then given by
\[
		\phi_i(\vec x) = \varphi_{k_{\ell i}}^{(\ell)}(\vec x)
.\] 

In this way, we can write $\phi_i$ as follows. 
Let 
\[
		\mathcal S_i = \{\ell : K_\ell \subset \supp \phi_i\}
.\] 
Then 
\[
		\phi_i(\vec x) = \sum_{\ell \in \mathcal S_i} 
		\varphi_{k_{\ell i}}^{(\ell)}(\vec x) 
		\mathbbm 1(\vec x \in K_\ell)
.\] 


\section{Assembling Matrix Forms}

To place the problem into its FEM discretized form, we 
build an approximation to $u$ of the form
\[
		u_h(\vec x) = \sum_{i=0}^{N-1} u_i \phi_i(\vec x)
.\] 
Let $V_h = H^1(\Omega)$.  This is the trial function space,
where $u_h$ will live.  
The test function space will be 
\[
		V_{h, D} = H_D^1(\Omega) = 
		\{v \in H^1(\Omega) : \restrict{v}{\Gamma_D} = 0\}
.\] 
We will then apply the variational form.  Doing so for 
$u \in H^1(\Omega), v \in H_D^1(\Omega)$ gives
\[
		\int_\Omega \nabla v \cdot \nabla u \, d^2 x + 
		c \int_\Omega v u \, d^2 x - 
		\int_{\Gamma_N} v g \, d\sigma = 
		\int_\Omega v f \, d^2 x
.\] 
We next use the approximation for $u$:
\[
		\sum_{i=0}^{N-1} u_i \int_\Omega \nabla v \cdot \nabla \phi_i \, d^2 x
		+ c \sum_{i=0}^{N-1} u_i \int_\Omega v \phi_i \, d^2 x = 
		\int_\Omega v f \, d^2 x + 
		\int_{\Gamma_N} v g \, d\sigma
.\] 
Next, since $v \in H^1(\Omega)$ we must require that 
for $0 \leq j \leq N-1$ that 
\[
		\sum_{i=0}^{N-1} u_i \int_\Omega \nabla \phi_j \cdot 
		\nabla \phi_i \, d^2 x + 
		c \sum_{i=0}^{N-1} u_i \int_\Omega \phi_j \phi_i \, d^2 x = 
		\int_\Omega \phi_j f \, d^2 x + 
		\int_{\Gamma_N} \phi_j g \, d\sigma
.\] 
Thus it is imperitive that we compute the following matrices and vectors.
\begin{itemize}
  \item Mass matrix, $M$:
		  \[
				  M_{ij} = \int_\Omega \nabla \phi_i \cdot \nabla \phi_j 
				  \, d^2 x 
		  .\] 
  \item Stiffness matrix, $S$:
		  \[
				  S_{ij} = \int_\Omega \phi_i \phi_j \, d^2 x 
		  .\] 
  \item Load vector, $\vec b$: 
		  \[
		    b_i = \int_\Omega \phi_i f \, d^2 x
		  .\] 
  \item Traction vector, $\vec t$:
		  \[
				  t_i = \int_{\Gamma_N} \phi_i g \, d\sigma
		  .\] 
\end{itemize}

\subsection{Mass Matrix}

For each entry we need to compute
\[
		M_{ij} = \int_\Omega \nabla \phi_i \nabla \phi_j \, d^2 x
.\] 
Using the fact that the $\phi_i$ are constructed from the $\varphi_j$:
\[
		M_{ij} = \sum_{\ell \in \mathcal S_i} \sum_{m \in \mathcal S_j}
		\int_\Omega \nabla \varphi_{k_{\ell i}}^{(\ell)} \cdot 
		\nabla \varphi_{k_{m j}}^{(m)} 
		\mathbbm 1(\vec x \in K_\ell) \mathbbm 1(\vec x \in K_m) \, d^2 x
.\] 
Simplifying the indicator functions:
\[
		M_{ij} = \sum_{\substack{\ell \in \mathcal S_i \\ 
		m \in \mathcal S_j}} \int_\Omega 
		\nabla \varphi_{k_{\ell i}}^{(\ell)} \cdot 
		\nabla \varphi_{k_{m j}}^{(m)} 
		\mathbbm 1(\vec x \in K_\ell \cap K_m) \, d^2 x
.\] 
Thus $\ell = m$.  That is
\[
		M_{ij} = \sum_{\ell \in \mathcal S_i \cap \mathcal S_j} 
		\int_\Omega \nabla \varphi_{k_{\ell i}}^{(\ell)} \cdot 
		\nabla \varphi_{k_{\ell j}}^{(\ell)} \mathbbm 1(\vec x \in K_\ell) 
		\, d^2 x = 
		\sum_{\ell \in \mathcal S_i \cap \mathcal S_j} 
		\int_{K_\ell} \nabla \varphi_{k_{\ell i}}^{(\ell)} \cdot
		\nabla \varphi_{k_{\ell j}}^{(\ell)} \, d^2 x
.\] 
Firstly, we need a method to compute $\mathcal S_i$.  
This can be done using the lookup table of elements which 
indicates which coordinates are a part of each element.  
So given a node $\vec z_i$, find elements which contain $\vec z_i$
as a vertex.  These are the elements of $\mathcal S_i$.  

It should also be noted:
\[
		\nabla \varphi_p^{(\ell)}(\vec x) = 
		\nabla \hat \varphi_p(\vec F^{-1}_\ell(\vec x)) = 
		\left(B_\ell^{-1}\right)^T 
		\hat \nabla \hat \varphi_p(\vec F^{-1}_\ell(\vec x))
.\] 
Therefore
\[
		M_{ij} = \sum_{\ell \in \mathcal S_i \cap \mathcal S_j} 
		\int_{K_\ell} \left[\left(B_\ell^{-1}\right)^T 
				\hat\nabla \hat\varphi_{k_{\ell i}}(\vec F_\ell^{-1}(\vec x))
		\right] \cdot 
		\left[\left(B_\ell^{-1}\right)^T
				\hat\nabla \hat\varphi_{k_{\ell j}}(\vec F_\ell^{-1}(\vec x))
		\right] \, d^2 x
.\] 
Finally, making the transformation
\[
		\vec x = \vec F_\ell(\hat{\vec x}) \implies 
		d^2 x = |\det B_\ell| d^2 \hat x
\] 
simplifies the integral to become
\[
		M_{ij} = \sum_{\ell \in \mathcal S_i \cap \mathcal S_j} 
		\int_{\hat K} \left[\left(B_\ell^{-1}\right)^T 
				\hat\nabla \hat\varphi_{k_{\ell i}}(\hat{\vec x})
				\right] \cdot \left[\left(B_\ell^{-1}\right)^T
		\hat\nabla \hat\varphi_{k_{\ell j}}(\hat{\vec x}) \right] 
		|\det B_\ell| \, d^2 \hat x
.\] 
For continuous piecewise linear functions, the integrand is a constant,
giving
\[
		M_{ij} = |\hat K| \sum_{\ell \in \mathcal S_i \cap \mathcal S_j} 
		|\det B_\ell| \left(\hat\nabla \hat\varphi_{k_{\ell i}}\right)^T 
		B_\ell^{-1} \left(B_\ell^{-1}\right)^T 
		\hat\nabla \hat\varphi_{k_{\ell j}}
.\] 
\subsection{Stiffness Matrix}

An analygous argument to that of the mass matrix will yield:
\[
		S_{ij} = \sum_{\ell \in \mathcal S_i \cap \mathcal S_j} 
		\int_{K_\ell} \varphi_{k_{\ell i}}^{(\ell)} (\vec x) 
		\varphi_{k_{\ell j}}^{(\ell)} (\vec x) \, d^2 x
.\] 
We can then substitute in terms of $\hat\varphi$ in the same way
and get
\[
		S_{ij} = \sum_{\ell \in \mathcal S_i \cap \mathcal S_j} 
		\int_{\hat K} \hat\varphi_{k_{\ell i}}(\hat{\vec x}) 
		\hat\varphi_{k_{\ell j}}(\hat{\vec x}) 
		|\det B_\ell| \, d^2 \hat x = 
		\sum_{\ell \in \mathcal S_i \cap \mathcal S_j} 
		|\det B_\ell| \int_{\hat K} 
		\hat\varphi_{k_{\ell i}}(\hat{\vec x}) 
		\hat\varphi_{k_{\ell j}}(\hat{\vec x}) \, d^2 \hat x
.\] 
We can build a lookup table:
\[
		L_{mp} = \int_{\hat K} 
		\hat\varphi_m(\hat{\vec x}) \hat\varphi_p(\hat{\vec x}) \, d^2 \hat x,
		\quad m, p = 0, 1, 2
.\] 
Which requires only 6 unique integrals.  Then
\[
		S_{ij} = \sum_{\ell \in \mathcal S_i \cap \mathcal S_j} 
		|\det B_\ell| L_{k_{\ell i} k_{\ell j}}
.\] 

\subsection{Load Vector}

We can employ a cheat with the load vector.  
First replace $f$ by its interpolant built out of the $\phi_i$:
\[
		I f := \sum_{i=0}^{N-1} f(\vec z_i) \phi_i
.\] 
Then
\[
		b_i = \sum_{j=0}^{N-1} f(\vec z_j) \int_\Omega \phi_i \phi_j \, 
		d^2 x = 
		(S \vec f)_i
.\] 
That is $\vec b = S \vec f$ where 
$f_i = f(\vec z_i)$.
While this is not exact, it drastically saves computational cost,
and the method is only accurate to the first derivative anyway.

\subsection{Traction Vector}

For the construction of the traction vector, first 
consider the set of indices $\mathcal S^{(N)}$ where 
\[
		\mathcal S^{(N)} := \{
				0 \leq \ell \leq N - 1 : 
		\vec z_\ell \in \Gamma_N\}
.\] 
We can further regard $\Gamma_N$ as a union of edges 
$E^{(N)}_n$ where
\[
		E^{(N)}_n = \{\vec x = t \vec z_{\ell_{0, n}^{(N)}} + 
		(1 - t) \vec z_{\ell_{1, n}^{(N)}} : 0 \leq t \leq 1\} \implies 
		\Gamma_N = \bigcup_{n=0}^{N^{(N)} - 1} E^{(N)_n
.\] 
Here 
the index $\ell_{i, n}^{(N)}$ for $i = 0, 1$ extracts the index so that 
the Neumann edges are numbered counterclockwise.  
Then the traction vector has the form
\[
		t_i = \sum_{n=0}^{N^{(N)} - 1} \int_{E_n^{(N)}} g \phi_i \, d\sigma
.\] 
However, it should be noted that for $t_i \neq 0$ we require 
$i \in \mathcal S^{(N)}$.  Let us assume this is the case. 
Then given a value for $i$, we can use the decomposition of $\phi_i$:
\[
		t_i = \sum_{n=0}^{N^{(N)}} \sum_{\ell \in \mathcal S_i} 
		\int_{E_n^{(N)}} g \varphi_{k_{\ell i}}^{(\ell)} 
		\mathbbm 1(\vec x \in K_\ell) \, d\sigma
.\] 
So for a non-zero contribution, we require 
$\vec x \in E_n^{(N)} \cap K_\ell$.

\newpage 

On $\Gamma_D$ the values of $u$ are known.  
Let $\mathcal I_D$ be the set of indices 
indicating 
\[
  \ell \in \mathcal I_D \implies \vec z_\ell \in \Gamma_D
.\] 
Then the sum can be split to examine only the free indices:
\[
		\sum_{\substack{i = 0 \\ i \not\in \mathcal I_D}}^{N-1}
		u_i \int_\Omega \nabla v \cdot \nabla \phi_i \, d^2 x + 
		c \sum_{\substack{i = 0 \\ i \not\in \mathcal I_D}}^{N-1}
		u_i \int_\Omega v \phi_i \, d^2 x = 
		\int_\Omega v f \, d^2 x + \int_{\Gamma_N} v g \, d\sigma 
		- \sum_{i \in \mathcal I_D} 
		u_D(\vec z_i) \int_\Omega 
.\] 
\end{document}
